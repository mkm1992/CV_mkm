 %%%%%%%%%%%%%%%%%%%%%%%%%%%%%%%%%%%%%%%%%
% "ModernCV" CV and Cover Letter
% LaTeX Template
% Version 1.11 (19/6/14)
%
% This template has been downloaded from:
% http://www.LaTeXTemplates.com
%
% Original author:
% Xavier Danaux (xdanaux@gmail.com)
%
% License:
% CC BY-NC-SA 3.0 (http://creativecommons.org/licenses/by-nc-sa/3.0/)
%
% Important note:
% This template requires the moderncv.cls and .sty files to be in the same 
% directory as this .tex file. These files provide the resume style and themes 
% used for structuring the document.
%
%%%%%%%%%%%%%%%%%%%%%%%%%%%%%%%%%%%%%%%%%

%----------------------------------------------------------------------------------------
%	PACKAGES AND OTHER DOCUMENT CONFIGURATIONS
%----------------------------------------------------------------------------------------

\documentclass[10pt,a4paper,sans]{moderncv} % Font sizes: 10, 11, or 12; paper sizes: a4paper, letterpaper, a5paper, legalpaper, executivepaper or landscape; font families: sans or roman

\moderncvstyle{banking} % CV theme - options include: 'casual' (default), 'classic', 'oldstyle' and 'banking'
\moderncvcolor{purple} % CV color - options include: 'blue' (default), 'orange', 'green', 'red', 'purple', 'grey' and 'black'

\usepackage{lipsum} % Used for inserting dummy 'Lorem ipsum' text into the template
\usepackage{multicol}

\usepackage{fontawesome}
\usepackage[scale=0.75]{geometry}%\usepackage{cmcyr}
 % Reduce document margins
%\setlength{\hintscolumnwidth}{3cm} % Uncomment to change the width of the dates column
%\setlength{\makecvtitlenamewidth}{10cm} % For the 'classic' style, uncomment to adjust the width of the space allocated to your name

%----------------------------------------------------------------------------------------
%	NAME AND CONTACT INFORMATION SECTION
%----------------------------------------------------------------------------------------

\firstname{Mojdeh } % Your first name
\familyname{Karbalaee Motalleb} % Your last name

% All information in this block is optional, comment out any lines you don't need
\title{Curriculum Vitae}
\address{}{\textbf{Tehran University-- Department of Electrical and Computer Engineering}}
\mobile{(+358) 414707755}
%\phone{(000) 111 1112}
%\fax{(000) 111 1113}
\email{ mojdeh.karbalaee@ut.ac.ir }
\homepage{skype-live:mkm1992} % The first argument is the url for the clickable link, the second argument is the url displayed in the template - this allows special characters to be displayed such as the tilde in this example
\extrainfo{mokm1992@gmail.com }
%\photo[70pt][0.2pt]{pictures/picture} % The first bracket is the picture height, the second is the thickness of the frame around the picture (0pt for no frame)
%\quote{"A witty and playful quotation" - John Smith}

%----------------------------------------------------------------------------------
%            content
%----------------------------------------------------------------------------------
\begin{document}

\makecvtitle
%----------------------------------------------------------------------------------------
%	EDUCATION SECTION
%----------------------------------------------------------------------------------------
%\section{Objective}

%Pursuing graduate studies in Power with an emphasis on Power Electronics and Power System towards a PhD degree and beyond so as to acquire sufficient knowledge and experience for a productive life time career in teaching and applied research.

%----------------------------------------------------------------------------------------
%	RESEARCH INTERESTS SECTION
%----------------------------------------------------------------------------------------
%%%%%%%%%%%%%%%%%%%%%%%%%%%%%%%%%5
%\section{Referee}
%\begin{itemize}
%\item Mr.Baharie
%\end{itemize}
%%%%%%%%%%%%%%%%%%%%%%%%%%%%%%%%%
%\section{Referee}
%Mr. Legzian

\section{Education}

\begin{itemize}
\item \textbf{Researcher} \hfill \textbf{March - since now}
\\
\emph{University of Oulu} \hfill \emph{Oulu -Finland}
\item \textbf{PhD} \hfill \textbf{2017--2022 (Expected)} \\
\emph{Tehran University} \hfill \emph{Tehran}
\begin{itemize}
\item \textbf{Major: Telecommunication Systems and Network}\hfill Total GPA: 18.85/20 via 19 credits
\end{itemize}
\item \textbf{Master of Science} \hfill \textbf{2015--2017 } \\
\emph{Amirkabir University of Technology} \hfill \emph{Tehran}
\begin{itemize}
\item \textbf{Major: Telecommunication Systems}\hfill Total GPA: 17.13/20 via 32 credits
\end{itemize}

\item \textbf{Bachelor of Science} \hfill \textbf{2011--2015} \\
\emph{Amirkabir University of Technology} \hfill \emph{Tehran}

\begin{itemize}
\item \textbf{Major: Telecommunication systems} \hfill Total GPA: 18.28/20 via 143 credits
\item \textbf{Minor: Mathematics} \hfill  Total GPA: 18.42/20  via 50 credits 
\end{itemize}

\item \textbf{High School} \hfill \textbf{2007--2011} \\
\emph{Farzanegan1(NODET)} \hfill \emph{Tehran}
\begin{itemize}
\item \textbf{Major: Mathematics and Physics} \hfill Total GPA: 19.73/20 via 20 credits
\end{itemize}
\end{itemize}


%----------------------------------------------------------------------------------------
%	HONOR SECTION
%----------------------------------------------------------------------------------------
\section{Honors}

\begin{itemize}
\item \textbf{Ranked 7\textsuperscript{st}} in Electrical Engineering,\textbf{Ranked 5\textsuperscript{st}} in Communication Group, among more than 35 students,
Amirkabir University of Technology, Tehran, Iran [Fall 2011]
\item \textbf{Ranked 12\textsuperscript{st}} in Olympiad of Electrical Engineering
\item \textbf {Acceptd}  for Internship as a researcher at Imperial College of London
\item \textbf{Ranked 412\textsuperscript{st}} in university entrance exam (Konkour), among more than 300,000
participant [Summer 2011]
\item \textbf{Exempted} from university entrance exam for M.Sc. program and offered M.Sc. program in Communication System in \textbf{Amirkabir} University of Technology
\item \textbf{Permitted to study Mathematics as a minor} (This permission is only awarded to talented students, introduced by the Exceptional Talents Office
\item Granted admission from \textbf{Talented Student Office} of Amirkabir University of Technology for graduate
study
\item Accepted to study at National Organization for Development of Exceptional Talents (\textbf{Nodet}) school
\end{itemize}

%%%%%%%%%%
\section{Research Interests}
\begin{multicols}{2} 
\begin{itemize}
%\item{Flexible AC transmission systems (FACTS)}
%\item{HVDC}
%\item{Power Quality}
%\item{Computer programming}
\item{Machine learning}
\item{Deep Learning}
\item{Data Science}
\item{Data Analysis}
\item{Reinforcement learning}
\item{Backend Developer}
\item{Telecommunication}
\item{Wireless Systems}
\item{Resource Allocation}

%\item{IoT-Sigfox, LoRa, NB-IoT}
%\item{Power System Optimization}
%\item{Distributed Generation}
\end{itemize}
\end{multicols}


%----------------------------------------------------------------------------------------
%	BEST COURSES SECTION
%----------------------------------------------------------------------------------------
\section{Some Courses}

\begin{multicols}{2}  % makes the text in two columns
\begin{itemize}
\item \textcolor{gray}{Bachelor Courses}
\begin{itemize}
\item Basic Computer Programming            \hfill{20}  
\item{Advanced Systems Programming}			\hfill{19}
\item Physics 1 \hfill{17.5}
\item Physics 2 \hfill{18}
\item Mathematics 1 \hfill{18.5}
\item Mathematics 2 \hfill{19.2}
\item Laser Electronics \hfill{18.25}
\end{itemize}
\item \textcolor{gray}{Master and PHD Courses}
\begin{itemize}
\item Statistical Learning \hfill{16.6}
\item Introduction to CryptoCurrency  \hfill{18.25}
\item Convex Optimization \hfill{19.1}
\item Cellular Network \hfill{20}
\item Data Network \hfill{19}
\item Neural Network and Deep Learning  \hfill{19}
\item Coding \hfill{17}
\item Resource Allocation \hfill{16.86}
\item Digital Signal Processing                                \hfill{17.7} 
\item Advanced Digital Signal Processing                                \hfill{18} 
\item Stochastic Processing         \hfill{18.11}
\item Broad band         \hfill{17.10} 
\item Engineering Mathematics           \hfill{18.6} 
\item Information Theory       \hfill{17.5} 
\item Linear Algebra \hfill{Audit}
\item Reinforcement Learning  \hfill{Audit}
\item Advanced Algorithms \hfill{Audit}
\end{itemize}
\end{itemize}
\end{multicols}

%----------------------------------------------------------------------------------------
%	TEACHING EXPERIENCE SECTION
%----------------------------------------------------------------------------------------


%----------------------------------------------------------------------------------------
%	PUBLICATION SECTION
%----------------------------------------------------------------------------------------


%----------------------------------------------------------------------------------------
%	ACADEMIC PROJECTS SECTION
%----------------------------------------------------------------------------------------
\section{Projects}

\begin{itemize}
 \item Testing (Throughput, trace-route, ...) of different protocols such as ICMP(using RAW socket), TCP, UDP,... 
 \begin{itemize}
     \item \textcolor{gray} {Implementing by C++ and python in ubuntu}
 \end{itemize}
\item Deep q-learning for solving resource allocation in Open-RAN system using network slicing
\begin{itemize}
\item \textcolor{gray}{Implementing Deep learning methods for Deep Q-learning using LSTM}
\end{itemize}
\item Design radar system with matlab for academic project
\item Design channelize receiver with matlab 
\item Digital Signal Processing for different digital modulation using Matlab and simulink 
\item Image Classification using Resnet and Efficient net for sorters
\item Determine Color Palette and Clustering Main Colors of any image
\begin{itemize}
\item \textcolor{gray}{Implementing Unsupervised Learning Using Python(opencv,skitilearn, PIL) and Javascript(canvas) [Winter 2019]}
\end{itemize}
\item Object Segmentation 
\begin{itemize}
\item \textcolor{gray}{Implementing Deep learning methods such as RCNN methods, Resnet, Using Python (Keras and Tensorflow) }
\end{itemize}
\item Edge Detection and image processing
\begin{itemize}
\item \textcolor{gray}{Using Python (Opencv and PIL)}
\end{itemize}
\item Processing a narrow band IoT protocol using SDR dongle \begin{itemize}
\item \textcolor{gray}{Obtaining different layer of protocol using Matlab and Simulink}
\end{itemize}
\item Simulation of Pendulum Waves by C++
\item Coding , Modulating and Transmitting Sound, PM Modulation, Simulating With Noise and Recieving,Demodulating and Decoding: Communication Systems 2 Project, Simulated by Matlab 
\item Simulation of Sound Wave when we have absorbant and obstacles by Python in Qt Designer
\item Simulation of indoor localization system with Access Point Selection and Signal Reconstruction with Matlab
\item Simulation of  Communication System with Matlab
\item Simulation of Precoding and detection in  Multi User MIMO 
\item Resource Allocation for CRAN system 
\item Simulation of Narrow Band System using SDR dongle as a receiver with Matlab 
\item Comparing different Standard of IoT 
\end{itemize}

%----------------------------------------------------------------------------------------
%	INTERNSHIP SECTION
%----------------------------------------------------------------------------------------
\section{Publications}
\begin{itemize}
\item
Resource Allocation in an Open RAN System using Network Slicing
\begin{itemize}
\item
\textcolor{gray}{
M. K. Motalleb, V. Shah-Mansouri, S. Parsaeefard and O. L. A. López, "Resource Allocation in an Open RAN System using Network Slicing," in IEEE Transactions on Network and Service Management, 2022 }
\item \textcolor{blue}{\href{https://ieeexplore.ieee.org/document/9888767}{IEEE link}}
\end{itemize}
\item
Joint Power Allocation and Network Slicing in an End-to-End ORAN System
\begin{itemize}
\item
\textcolor{gray}{
M karbalaee motalleb, v shahmansouri, s nouri, "Joint Power Allocation and Network Slicing in an End-to-End ORAN System ," 2019 }
\item \textcolor{blue}{\href{https://arxiv.org/pdf/1911.01904.pdf}{Arxiv link}}
\end{itemize}
\item
Optimal Power Allocation for Distributed MIMO C-RAN System with Limited Fronthaul Capacity
\begin{itemize}
\item
\textcolor{gray}{
m karbalaee motalleb, a kabiri, mj emadi, "Optimal Power Allocation for Distributed MIMO C-RAN System with Limited Fronthaul Capacity," in  ICEE 2017 }
\item \textcolor{blue}{\href{https://ieeexplore.ieee.org/abstract/document/7985380}{IEEE link}}
\end{itemize}

\end{itemize}
%\end{itemize}

\section{Academic projects}
\begin{itemize}
\item B.Sc project
Under the supervision of Dr.Emadi 
\begin{itemize}
\item On the Capacity of Molecular Communication over the AIGN Channel

\end{itemize}

\item{M.Sc project}
Under the supervision of Dr.Emadi
\begin{itemize}
\item Distributed cooperation to enhance performance of Cloud Radio Access Netework

\end{itemize}
\item{PhD project}
Under the supervision of Dr.Shahmansouri
\begin{itemize}
\item Resource Allocation using Network Slicing in the Open Ran system
\end{itemize}
\end{itemize}

\section{ Work Experience}
\begin{itemize}
\item Working as a Software Tester at Sina Company( Spring 2021 Since Spring 2022)
\begin{itemize}
\item
\textcolor{gray}{I was a software developer and machine learning engineer at Sina Company. I was on the Test Team. I was writing different test scenarios for fiber optic systems (OTN systems) and testing the system using
python and pytest. We have done test scenario for physical layer too.
Also, I was a python developer here, and we used Flask for Rest Api of the backend system.}
\end{itemize}
\item Consulting as a data scientist for Nojan Company( Spring 2021)
\begin{itemize}
\item
\textcolor{gray}{I was a data scientist consulter at Nojan company. I have developed image classification and image segmentation. I have done transfer learning using a pre-trained model for imagenet dataset and methods such as resnet, effieicnt net, and other models to cluster pistachio's images to classify them based on their shape and quality.}
\end{itemize}
\item Work in Sepehran Company as a  researcher (Summer 2019 since Fall 2020)
\begin{itemize}
\item
\textcolor{gray}{I was in the Research and Development department of Sepehran company. I have done signal processing projects using filter banks for 
high-frequency signals. Also, I have designed an LFM Radar system for the industry. I used Matlab and Simulink to develop my code. Moreover, I used System Generator to convert them to the VHDL. 
 Furthermore, we used machine learning methods on different digital modulation to cluster signals based on their modulation using MATLAB,  Simulink, and python.}
\end{itemize}

\item Work in RMI Company as a backend developer (Fall 2018 since Spring 2019)
\begin{itemize}
\item
\textcolor{gray}{I was a python developer in this company. I have done image processing and machine and deep learning projects such as segmentation an image by CNN methods and clustering the image based on color
palettes. This application uses deep learning methods such as res-net, efficient-net, and other methods to develop image segmentation. Then we use different image processing approaches, filtering, and clustering techniques to find the color palette of rugs and carpets for carpet companies. We developed code in python using PyTorch. Moreover, we use Flask, and Tornado for Rest Api and socket programming.}
\end{itemize}


\item Work in Parsnet Company as a  researcher (Fall 2017 since Fall 2018)
\begin{itemize}
\item
\textcolor{gray}{I was in the Research and Development department of Parsnet company. I have done signal processing projects on LPWAN signals such as Sigfox and LoRa using RTL SDR dongle and USRP. We work on enhancing LoRa and Sigfox protocol and find their weakness. Moreover, We have developed indoor positioning projects for Wireless sensor networks using machine learning methods using the SVM method. We develop our codes in Matlab and C++.}
\end{itemize}
\end{itemize}
%---------------------------------------------------------------------------------------
%----------------------------------------------------------------------------------------
%	LANGUAGE SKILLS SECTION
%----------------------------------------------------------------------------------------
\section{Teaching Experience}

\begin{itemize}
\item Teaching Assistant for \textbf{Computer Programming} Undergraduate Course \hfill \textbf{Fall 2013}

\item Teaching Assistant for \textbf{Numerical Analysis} Undergraduate Course\hfill \textbf{Spring 2014}

\item Teaching Assistant for \textbf{Computer Programming} Undergraduate Course\hfill \textbf{Fall 2015}

\item Teaching Assistant for \textbf{Communication I} Undergraduate Course\hfill \textbf{Fall 2015}

\item Teaching Assistant for \textbf{Advanced Programming} Undergraduate Course\hfill \textbf{Spring 2016}

\item Teaching Assistant for \textbf{Computer Programming} Undergraduate Course\hfill \textbf{Fall 2016}

\item Teaching of \textbf{Software Defined Radio Lab with MATLAB} Undergraduate Course\hfill \textbf{Fall 2018}

\item Teaching of \textbf{Service Based Architecture with python} graduate Course\hfill \textbf{Spring 2021}

\item Teaching (Private) for \textbf{Mathematics} 
\end{itemize}


%	COMPUTER SKILLS SECTION
%----------------------------------------------------------------------------------------
\section{Computer skills}
\begin{cvcolumns}
  \cvcolumn[0.35]{Programming Languages}{\begin{itemize}
  \item MATLAB
  \begin{itemize}
      \item \textcolor{gray} {M-file}
      \item \textcolor{gray} {Simulink}
      \item \textcolor{gray} {System Generator(familiar)}
      \item \textcolor{gray} {CVX}
  \end{itemize}
  \item C++ 
  \item Robot Framework 
  \item SQL
  \item Python \begin{multicols}{2} \begin{itemize}
\item \textcolor{gray}{Keras}
\item \textcolor{gray}{Tensorflow}
\item \textcolor{gray}{Pytorch}
\item \textcolor{gray}{gym}
\item \textcolor{gray}{Opencv}
\item \textcolor{gray}{PIL}
\item \textcolor{gray}{Tornado}
\item \textcolor{gray}{Flask}
\item \textcolor{gray}{Django(familar)} 
\item \textcolor{gray}{pytest}
\item \textcolor{gray}{Seaborn}
\item \textcolor{gray}{Folium}
\item \textcolor{gray}{Bokeh}
\item \textcolor{gray}{Numba}
\item \textcolor{gray}{Numpy}
\item \textcolor{gray}{Matplotlib}
\item \textcolor{gray}{Cython}
\item \textcolor{gray}{Sklearn}
\item \textcolor{gray}{Skimage}
\item \textcolor{gray}{Pandas}
\item \textcolor{gray}{Os}
\end{itemize}
\end{multicols}
   \item R (familiar) \item Javascript \item Node js(familiar) \end{itemize}}
  \cvcolumn[0.35]{Software tools}{\begin{itemize}\item Git \item \LaTeX  \item Qt Designer(familiar) \item GNU Radio using SDR dongle(familiar) \end{itemize}}
  \cvcolumn{O.S and General Softwares}{\begin{itemize} \item Ubuntu \item Mac \item Microsoft Windows \item Microsoft Office \end{itemize}}
\end{cvcolumns}


%----------------------------------------------------------------------------------------
%	HOBBIES AND INTERESTS SECTION
%----------------------------------------------------------------------------------------
\section{Language Skills}

\begin{itemize}
\item \textbf{Persian} \hspace{5 pt} Native
\item \textbf{English} \hspace{6 pt} Fluent   
\item \textbf{French} \hspace{6 pt} Elementary  % \textcolor{gray}{Just start learning}
%\item \textbf{Arabic} \hspace{10 pt} Familiar \\
\end{itemize} 
\section{Hobby}
\begin{multicols}{2} 
\begin{itemize}
\item Cycling
\item Playing Guitar
\item Studying English and French
\item Solving geometric problems
\item Driving 
\item Swimming
\end{itemize}
\end{multicols}

%----------------------------------------------------------------------------------------
%	References and Proofs
%----------------------------------------------------------------------------------------

\centerline{\underline{\textbf{ \hspace{0.5 pt} References, Further information, and Proofs are available upon Request}}}

\end{document}

%----------------------------------------------------------------------------------------
%----------------------------------------------------------------------------------------